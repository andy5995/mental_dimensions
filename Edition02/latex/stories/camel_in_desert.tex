\chapter{The Camel in the Desert}

Once upon a time, on a world far from our own, in a desert that spanned thousands of kilometers, a camel walked near a narrow stream of water. The camel lowered his head to drink. By the time his mouth was near the ground, however, all the water had evaporated. He began walking again to find more. He'd hoped to find a lake, but even a small river would make him happy.

Being thirsty made him sad, but he was also sad because he was the only camel remaining. All his other camel friends had died. One had even committed suicide. He'd only been able to say goodbye to one of them; the others had gone up to Heaven much too suddenly.

He walked along further, and in the distance saw a lizard. The lizard wasn't moving, and the camel approached the lizard easily. “Mr. Lizard, why aren't you moving?”

The lizard replied, “Oh, Mr. Camel, I'm so thirsty I can't even move my legs. I'm conserving my energy until it rains.”

The camel said, “But Mr. Lizard, that may not be for months. You may die before it rains. Let me help. I have water stored in my hump and will give you some.” The camel spit on the lizard, and onto the ground in front of the lizard. He drank some and felt much better.

“Now climb onto my back and you can rest until I find a small river, or hopefully a large lake.”

“Thank you, thank you, Mr. Camel!” The lizard slowly crawled onto the camel's foot, up its leg, around his stomach until he was resting comfortably upon the camel's back.

The two walked for many kilometers until they spotted a bird. The bird was very thirsty. After exchanging routine social pleasantries, the camel gave the bird some water, just as he had done for the lizard. He then invited the bird to perch on his tail while he continued his search for water.

Finally, they saw a small patch of grass, and beyond the grass, some green bushes. Beyond the bushes was a large lake. The three—who after going through such a life-threatening ordeal—were now good friends. They each drank from the lake, and ate some berries from the bush. They found the materials needed to build a rudimentary shelter.

It rained later that evening, and they knew they'd chosen a fine spot in which to stay. Soon, other creatures wandering through the desert found their little settlement. They expanded it over time, and after a few years it became a well-populated city. None of its citizens ever died of thirst because rain came regularly. It became a tourist attraction to some space-faring camels that landed there by accident when their spaceship ran out of ion particles, a rare fuel, which could be engineered using water and leaves from berry-bushes.