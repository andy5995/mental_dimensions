\chapter{The Obedient Parents}
\setlength{\parindent}{2em}
% \fontsize{16pt}{16.3pt}
\LARGE
\baselineskip=1.15\baselineskip

Once upon a time, in a land not far from yours, there lived a small boy and a small girl who sometimes did not like to listen to their parents. The boy's name was Billy and the girl's name was Holly. Sometimes the parents would say “It's time to go to bed, dear" or “You can't have ice cream until you eat your vegetables, honey." Often the parents had to order the children to clean their bedroom.

On this particular occasion, Billy and Holly were cleaning their room because their parents told them to do so. As they were picking up their toys, they each reached for a small plastic elf. When they touched it at the exact same time, it began to glow bright-green and made a sparkling kind of noise. The elf quickly grew to be as tall as Billy and Holly's teacher at school. He had a bright-orange mustache and beard. His ears were pointed and he was dressed in clothes that seemed to be about 5000 years old.

They looked at him in astonishment, as he said, “Congratulations, children. Because of your teamwork, you've won three wishes.”

“Yay!”, exclaimed Billy and Holly.

“Holly, what should we wish for?”

“I don't know Billy. I want so many things I can't think of what I want the most!”

“I know!”, shouted Billy. “Let's wish that mom and dad have to listen to us from now on. Let's wish they always have to do what we tell them.”

Holly replied excitedly, “That's a great idea, Billy!”

The elf said, “I don't think that's a good wish, but I'm willing to grant it.” Instantly, enchantment music and flashing multi-colored lights flooded their bedroom.

The music and the lights quickly vanished, however, and the elf shrank back to his three-inch toy form. At that moment, the parents came into the room. “Why aren't all your toys picked up?” The father asked.

Billy answered, “We forgot, Father. Please forgive us.”

The father's expression became blank, and—without emotion—he replied, “I forgive you.”

The mother was confused at the father's demeanor. She suggested, “Richard, why don't you go run your errands now? I'll talk to the children.” As she spoke, the father's expression turned back to normal.

“Okay, Elizabeth. I'll be home in about two hours.” The father turned to leave and began walking out of the room.

Billy yelled, “Stop!”

The father stopped, as if frozen in time. Billy and Holly looked at each other, mischievous grins carved on their faces. The mother said, “Richard, dear, is something the matter?”

“Nothing's the matter, Elizabeth. The children told me to stop, so I stopped.”

“But Richard, why are you obeying Billy and Holly? Is this some game you three invented without telling me? I don't think it's at all appropriate and--”

Holly interrupted, “Be quiet, Mother.”

The mother looked at them angrily, but said nothing.

“Mother, don't be angry with us.”

The mother smiled at Billy and Holly, and gave them a warm tender hug and kiss.

“Give us ice cream, Mother. Father, clean our room.” The parents did as they were told. The children were so happy they completely forgot about their remaining two wishes.

From then on, the parents were absolutely obedient to every whim and desire of their children. Billy and Holly decided for themselves when and what they'd eat, what time they'd go to sleep, and when they could play.

The parents would still tuck the children into bed at night, and give them love and hugs and kisses. However, if Billy and Holly commanded them to do or not do something, they had no choice but to obey.

And that's how life in their home went for the next seven years. It finally came to an end one day though, when the parents were cleaning the children's room. It was a rainy and cloudy Tuesday evening. Elizabeth accidentally dropped and broke Holly's favorite decoration—a lovely replica of Cinderella's glass slipper. She screamed at her parents, “Get out of here! I never want to see you again!”

The parents slowly turned around and walked out of the bedroom. They walked downstairs. They walked to the front door. They opened the front door and walked out to the driveway. They entered their car and pulled out of the driveway.

Billy heard what happened and met Holly in her room. “We have to get them back, Holly! We still need them!” Holly, having had a chance to calm down, agreed, and both rushed downstairs. They hurried outside and looked around. They saw their parents far off in the distance driving away in the family automobile. It was so tiny it looked like a little dot on the horizon. They shouted, “Come back!” But their cries were simply not loud enough for the parents to hear. They began running after the car, but quickly became short of breath and grew pains in their sides (due to a lack of exercise). A few moments later, the parents completely vanished from sight.

Billy and Holly both trudged back home. They laid down in bed, crying themselves to sleep. They had terrible nightmares. In the middle of the night they woke up soaked in sweat and tears. They wanted to crawl into bed with their parents, and get love and hugs and kisses, but quickly remembered their parents were gone forever. They went back to bed and cried again until morning. After the sun came up, they went to the kitchen to eat, for they were very hungry. But there were no clean dishes, and they had never learned how to wash dishes. Billy and Holly also never learned how to prepare a meal or operate an oven. Finally they found some dry, packaged food in the cupboard that required no preparation or cooking. They spent the remainder of the day trying to escape from their self-created nightmare and played in their room.

Every day they ate a little food from the cupboard, and every day they tried to pretend nothing was wrong. At the end of each day, however, they were weaker and thinner than the last. Their tummies started showing the outlines of their ribs. Their cheeks had lost all color, and drew in as they lost weight from undernourishment. Their eyes became dark and hollow. Without good food and love and hugs and kisses from their parents, they became ill. After several weeks they could barely get out of bed.

One morning, Holly was in bed feeling extremely weak but wanted to get some water for Billy. He hadn't opened his eyes for almost two days and he could only speak in short whispers. Holly was getting very worried. She reached under her bed for a glass, but her hand grabbed something else. It was a toy shaped like an elf. She picked it up, looked at it, and remembered.

She cried, “Oh magical elf, please help us!” Nothing happened. She crawled over to Billy and placed the toy in his hand. Suddenly, the magical elf reappeared. He looked down at the children with sadness in his eyes.

“I may be able to help you,” he said. “I hope your second wish is better than your first.”

“Oh...  I wish that our parents were back home.”

The elf replied, “That wish is granted.” Just then, Holly heard footsteps downstairs near the front door. “You have one wish left. I suggest you make it worthwhile.”

Holly thought hard. Her head was spinning and she fought back the urge to faint from severe hunger pains and thirst. She looked over at Billy with tears in her eyes. “Elf, I wish we'd never made our first wish.” As soon as the words were out of her mouth, sound effects and a rainbow of colored lights flooded the bedroom.

Billy and Holly were instantly as they were 7 years ago. Billy's eyes were open. Their cheeks were puffy and had good color. Their eyes were wide and bright with happiness and joy. They were healthy again. The elf transformed back into his small toy form. Even the bedroom looked as it did seven years ago. The parents walked into the room.

The father looked at them and said, “Why aren't all your toys picked up?”

Billy replied, “We're sorry, Father. We were playing with this elf and forgot.”

The mother said, “It's okay. Finish cleaning your room and come down to eat in a half-hour. We're having steak, eggs, carrots, and Brussels sprouts. For dessert we'll have some ice cream.”

“Okay!”

The parents smiled at their children and turned to leave the room. Out of the corner of his eye, however, the father spotted the toy elf. He walked over and picked it up. He smiled at the children and said, “You won't be needing this anymore.”

He met with Elizabeth in the kitchen, handed her the elf and said, “Do you think the children learned anything?”

“I think so. Did you notice how they didn't talk back when I told them we're having Brussels sprouts with our dinner?”

“Ah, yes, of course. Well, I'm sure they'll never forget, just as we never forgot. How old were we when you found this elf?”

“Five or six years old. I found it the day my family and I moved into the neighborhood next door to you and your family. You were arguing with them about doing dishes.”

“Yes, I recall that. We met the next day.  I'll never forget playing with you and that elf, and making that ridiculous wish about having obedient parents.”

The children finished cleaning their room and came downstairs to join their parents at the dinner table.