\chapter{A Shared Ending}

William lived in a small town a few miles from the city. In his youth, he had been an amateur boxer, but after two years, he joined the United States Navy, deciding it could offer better long-term benefits. When William had been in the Navy for about two years, a shipmate introduced him to a woman named Dorothy, who was a singer touring with the USO.

While in the service, William only was able to see Dorothy twice a year. They usually went dancing, or walking along a sandy shore of an ocean. During that time, they also kept in touch by mail.

Six years later, William felt he'd seen enough of the world and wanted to stay in one place. He bought a house in a rural area that was about a thirty-minute drive from where Dorothy lived. They continued seeing each other and married after a year.

During that year, he became employed as a supervisor for a company that manufactured gas-powered engines. His experience in the Navy, coupled with his honorable discharge, ensured that he was able to avoid starting at “the bottom”; on his first day of work, he began training for a position as supervisor. He learned quickly and soon became an excellent team leader. After eight years, he was promoted to assistant manager of the entire manufacturing facility.

William was over six feet tall, which meant that his co-workers always fought to have him on their team when they met after work for basketball. Several of William's female co-workers were drawn to his blue eyes, well-built physique, and good manners. They often flirted with him, but William always clearly expressed that the only woman who would ever hold his interest was Dorothy. He always added, “And that's a fact that will never change.”

William retired at the age of sixty-eight. He spent the next ten years making sure Dorothy was happy, keeping up with repairs on their four-bedroom house, and graciously accepting his role as patriarch of the family; which by this time included his two sons and two daughters, ten grandchildren, and various in-laws. In the past, they often came to visit William and Dorothy, but over time they became busier with their own family and careers, and visits became a rare, albeit happy, event.

A few days after his seventy-sixth birthday, William was told by his doctor, “You're generally in good health, but I recommend that you start using a cane when you walk. Also, be more careful what you eat—your heart seems like it's working a bit harder than it needs to for someone your age.” William followed his doctor's advice, and Dorothy made a few changes to her recipes when she cooked their meals.

Two years later, after complications from an illness, Dorothy passed away. “It was her time,” the doctor said. William disagreed but knew that he couldn't do anything about it except thank the Lord for their time together. Their family came out for the funeral. Afterward, William was alone, with only his memories.

He eventually accepted Dorothy's passing, but still missed her terribly. Visits from his children and grandchildren came less often due to their busy schedules. Most of William's friends had either died or moved away. He was on good terms with his neighbors but had little in common with them, so the topics of their conversations were often limited to the weather and current events. He began to feel lonely, which was something he hadn't felt since before he met Dorothy.

William was unable to drive a car due to the deterioration of his eyesight, so he had limited options for going out and meeting people his own age. He very much longed for the companionship of people from his generation—those who enjoyed the same music, were fans of the same movie stars, remembered the same major historical events, and so forth.

On William's eighty-first birthday, his oldest daughter and two grandchildren came to visit for three days. They all had a good time, but after they left, William felt even more alone than before. He decided to go for a walk, hoping to bump into some of his neighbors. They were often away at work or busy taking care of their fields, but sometimes he found one or two who could take a break and were interested in a short afternoon bull session. William grabbed his cane and went outside.

He soon reached the main road near his house, crossed it, and walked along the left side of it heading north. The sun was peeking behind some large, fluffy clouds. From the trees nearby he heard birds singing. He breathed deeply and looked up to see the clouds parting to make room for more sunlight.

William walked for about ten minutes but didn't see any neighbors or anyone else out walking. He decided to sit and rest on the old, rusty iron bench a few yards ahead. As he neared it, because a large oak tree obstructed his view, he could only see a corner of the bench. As he walked around the tree though, he saw a woman sitting on the bench. He'd never seen her before, which surprised him because he thought he knew everyone who lived nearby.

As he approached, they made eye contact. William greeted her. “Good afternoon, ma'am.”

“Good afternoon,” she returned with a smile. She looked up at the sky, then back at William. “It's such a beautiful day!”

William leaned on his cane more; it eased the pressure on his leg and hip. “Yes, it is. My name's William. I don't recall meeting you before.”

“I'm Mary. I moved into the city a few months ago. I was feeling a bit crowded so I thought I'd go for a walk. It's nice to meet you, William.”

“Same here.” He sat down next to Mary—a respectable twelve inches between them–and laid his cane against the bench's armrest.

William and Mary continued talking about the weather, then switched the topic to local politics, then to global current events. Before long, they discovered several things they had in common: they liked the same movies, the same music, and voted for the same presidents in the last seven elections.

After an hour had passed, they made plans to meet at the bench the next day.

As they were parting, William said, “It was a real treat to meet you, Mary. Tomorrow I'll bring a thermos with some water.”

“That's a good idea. I didn't plan to be out this long and I sure did get thirsty. I'll bring some chocolate-covered pretzels for us to snack on.”

“That's an even better idea!” They both chuckled, then they smiled, looking forward to coating their tongues with the taste of chocolate.



☼ ☼ ☼





Each day for seven days, William and Mary agreed to meet each other the following day. On the eighth day, William suggested, “As you know, Mary, my house feels empty with only me in it. I'd like to invite you to come stay with me for two weeks.” He gently took her hand.

Before answering, she turned her head and coughed. She cleared her throat, smiled at William and said, “I think that's a wonderful idea! I'll bring a small suitcase with some clothes and other odds and ends tomorrow. That should be all I need.”

“That cough doesn't sound too good. Are you okay?”

“Yes, I'm fine.” Mary answered nonchalantly, and added, “Just a little cold I think.” They hugged each other and then parted.

The following afternoon, they again met at the bench, but now Mary had a pink leather suitcase. William offered to carry it for her, but feeling that his cane was burden enough for him, she politely declined.

When they reached his house, he led her through each room, telling her about a special memory from each one. Arriving at the last room, he said, “This is where you'll be sleeping. My room is right next door.”

Mary looked at the two paintings on the wall, the light blue curtains over the window, and finally at the pictures of his family. “This room is simply lovely!”

“Thank you. Do you think you'll be comfortable here?”

Mary noticed the bed and replied, “The bed is the same kind as the one I have at home. I'm sure I'll be just fine, thank you, William.”

They ate supper together that evening. Afterward, they played cards, talked, and watched television. At around ten o'clock they both became tired and went to their rooms for a good night's sleep.

The next three days were fun for them both. Sharing stories from their childhoods, listening to music, and going for walks were among the many activities they did together. William didn't feel alone anymore, and Mary no longer felt crowded. They both felt blessed; however, they knew that their time together wouldn't last forever.



☼ ☼ ☼



Mary woke up feeling short of breath. She knocked on William's door. He got up to open it. He looked in her eyes, heard her breath, and knew she was ill. With concern evident in his voice, William asked, “What's wrong?”

“I'm just a little short of breath—and feeling a little dizzy.”

“You should go back to bed. I'll get you some water.”

“No, thank you, William, I'm just not thirsty right now.” He lightly stroked her hair as she went back into her room to lay down.

“I think we should go to the hospital, Mary. One of my neighbors told me I could call him if I ever had an emergency. He would drive us, and I know he's home right now.“

Mary chuckled, “Oh, William, don't worry! This happens sometimes. My doctor said that when it does, I only need to rest. If I start to feel worse, you'll be the first to know,” Mary said with a smile—but shakily, her weakened condition apparent in her voice.

“All right. You just rest then, and don't worry about how long you need to stay here. I'll be nearby if you need anything.” William sat down in a chair. Mary, now in bed, slowly turned over to look out the window. From his back pocket, William pulled out a crossword puzzle. He couldn't concentrate on it though; he continually glanced at Mary so he'd know right away if he should take her to a hospital. He watched for any signs that indicated her condition was worsening.

Soon, William heard Mary lightly snoring. Seeing how peaceful she was, he felt more relaxed and let his head rest on the back of the chair. Within a minute, he too had drifted peacefully into sleep; however, three times he awoke, looked at Mary, and each time fell back asleep.

The fourth time he awoke, he saw that Mary had turned back over and now faced him. She slowly opened her eyes, and saw William sitting nearby, looking as if he were about to fall asleep.

“William, come lie next to me and hold me.” With some effort, she moved to the other side of the bed to make room for him. He slowly rose from his chair and lay next to Mary. He felt weaker and more tired than usual, and his chest felt tight.

Before falling asleep, William's happiness caused a great smile to form on his face. Mary, with her friend's arms around her, felt very relaxed and happy; she, too, grew a broad smile before drifting peacefully to sleep.



☼ ☼ ☼





The next day, a police car with two officers inside drove into William's driveway. They parked but left the vehicle running.

“You sure this is the right address, Mitch?”

Mitch checked his notebook and compared it to the number on the front of the house. “Yeah, this is it.”

Steve inquired, “Who called it in?”

“Lawrence Middleton, William's son. After trying to phone his father and getting no answer, Lawrence took the first flight out and arrived in the city this morning. He rented a car and got here an hour ago. Found his father lying on the bed. He said that he checked for a pulse but couldn't find one. He left, drove to a restaurant and called the precinct. He figured that heart failure, combined with old age, was the cause of death.”

Steve sighed sadly. “All right. Let's check it out.” They exited the vehicle and walked up the steps to the front door. Mitch knocked once and waited fifteen seconds before trying the doorbell. Another fifteen seconds passed, so Mitch looked at Steve, expecting him to agree that they had waited long enough before entering the owner's residence. Steve nodded his consent. Mitch opened the door and they walked inside. They proceeded to look into each room, and in the fourth room they found William laying motionless on the bed.

Mitch sat down in the chair next to the bed, leaned toward William and checked him for a pulse. He lifted William's right eyelid and observed that the pupil failed to contract. “He's dead. No apparent cause but that's for the coroner to figure out.” Mitch respectfully pulled a sheet over William's face and the top of his head.

“Yeah, Stan'll figure it out,” Steve agreed. “Busy day for him. He's got that other one too.”

Mitch said, “Usually they don't have a smile on their face. I wonder what he was thinking about when he died.”

“I dunno. It's weird though—like that lady I found earlier this morning.”

“I hope I have a smile on my face when I go.”

Steve responded, “I don't like to think about it. Roger and I found a woman dead this morning in her home near Fifth and Park Avenue.” Steve stopped and thought for a moment, trying to remember a name, then continued, “Mary was laying in bed with a smile on her face. On the coffee table in her living was a suitcase that was open and half-packed. There were blood stains on her clothes; because of the tissues we found on her floor, Roger figured it was from a coughing fit Mary had before she died. In spite of that, she had a big smile on her face, just like this guy does.”

“That's a shame—she didn't even get to finish packing her suitcase. I hope she called ahead to wherever she was going.”

“That's not funny, Mitch.”

“Wasn't meant to be.”